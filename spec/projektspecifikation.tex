\documentclass{scrartcl}
\usepackage[utf8]{inputenc}
\usepackage[swedish]{babel}
\usepackage{booktabs}
\usepackage{longtable}
\usepackage{tabu}
\usepackage{caption}
\usepackage{wasysym}
\usepackage{environ}
\usepackage{array}

% utseende på labels
\captionsetup{margin=10pt,font=small,labelfont=bf,labelsep=period}

\author{\textbf{Magnus Johansson}\\
		magjo722@student.liu.se\\ \\
		\textbf{Rolf Lifvergren}\\
		rolli107@student.liu.se}

\title{Projektspecifikation}
\subtitle{Plattformsspel}

%% Kommando för kod inne i texten
%% Användningsexempel:
% \code{(define (square x) (* x x))}
\newcommand{\code}[1]%
{\texttt{#1}}

%% Kommando för ADT-tabell
%% Användningsexempel:
% \begin{adt-table}{\texttt{map\%}}{map-adt}
% 	render & canvas, dc & Ritar ut banan på skärmen
% \end{adt-table}
\NewEnviron{adt-table}[2]%
{\begin{longtable}{>{\bfseries}lcp{8cm}}
\caption{#1}\label{#2}\\

\toprule
\textbf{Metod} & \textbf{Argument} & \textbf{Beskrivning} \\ 
\midrule
\endfirsthead

\toprule
\multicolumn{3}{c}%
{\bfseries \tablename\ \thetable{}. (forts)} \\
\textbf{Metod} & \textbf{Argument} & \textbf{Beskrivning} \\ 
\midrule
\endhead

\multicolumn{3}{r}%
{(fortsätter på nästa sida)} \\
\endfoot

\bottomrule
\endlastfoot

\BODY
\end{longtable}}

\begin{document}
\maketitle
\clearpage

\section{Projektplanering}
Detta projekt går ut på att utveckla ett 2D-plattformsspel där spelaren ska hoppa, skjuta, undvika fiender och samla power-ups för att rädda världen (vilket man gör genom att besegra en mäktig boss).

\subsection{Spelhistoria}
Spelet är ett klassiskt 2D-plattformsspel där spelaren styr en karaktär som kan röra sig i en värld genom att springa och hoppa. I världen finns fiender och andra hinder (t.ex. spikar eller avgrunder) som måste undvikas eller besegras med hjälp av olika vapen. För att klara spelet måste en boss, som är den svåraste fienden i spelet, besegras.

Varje gång spelaren blir skadad så tappar denne hitpoints. När spelarens hitpoints tagit slut dör karaktären och spelaren förlorar ett liv. När liven tagit slut har spelaren förlorat och måste börja om. 

\subsection{Utvecklingsmetodik}\label{utvecklingsmetodik}
Det är tänkt att vi ska dela upp kodandet lika mellan oss (vilka delar specificeras senare) så att arbetet kan färdigställas snabbare. Vi har tänkt att använda oss av Git för versionshantering, till exempel med Astmatix som central server (vilket borde lösa problemet med att koden ska vara tillgänglig även om en gruppmedlem är sjuk eller borta). Vi kommunicerar med varandra via sms, mejl och Skype samt muntligt.

I första hand har vi tänkt att jobba under normal skoltid, dvs 8-17 på vardagar, men kommer även att arbeta på kvällar och helger i de fall tiden inte räcker till på dagarna. Att sitta tillsammans så ofta som möjligt när vi arbetar är önskvärt för att effektivt kunna dela idéer och samarbeta.

Vi räknar med att lägga 160 timmar totalt, vilket motsvarar de 3hp som projektet är på. Det betyder att vi ska jobba 80h/person, vilket i snitt blir 2{,}2 h/dag om vi jobbar 6 dagar i veckan.

\subsection{Grov tidplan}
I vår kravlista ses i vilken ordning vi tänker göra saker. Målsättningen är att krav av prioritet 1 skall göras direkt och prioritet 2 innan halvtidsmötet. Högre ordningens krav kommer efter halvtidsmötet.

Se kapitel \ref{utvecklingsmetodik} för preliminär tidsplan.

\subsection{Betygsambitioner}
Vi vill göra ett så bra projekt som möjligt på den tid vi hinner lägga på projektet och därför satsar vi högt, men eftersom duggorna har gått riktigt bra för oss båda är det inte avgörande för våra betyg i kursen hur högt vi når. Dock är vi båda väldigt intresserade av spelprogrammering så vi kommer med glädje att arbeta med projektet.

\section{Användarmanual}

\subsection{Kravlista}
Projektets kravlista återfinns i tabell \ref{kravlista} nedan.

\begin{table}[ht]
	\caption{Kravlista}\label{kravlista}
	\centering
	\begin{tabular}{lp{8cm}rl}
	\toprule
		\# & Beskrivning & Prioritet &  \\
	\midrule
	1 & Spelet ska kunna styras med tangentbordet &
		1 & $\CheckedBox$ \\
	1.A & Spelaren ska kunna hoppa &
		1 & $\CheckedBox$ \\
	1.B & Spelaren ska kunna gå i sidled &
		1 & $\CheckedBox$ \\
	2 & Fiender skall finnas och vara möjliga att interagera med &
		1 & $\CheckedBox$ \\
	3 & Vapen skall kunna användas för att eliminera fiender &
		1 & $\CheckedBox$ \\
	4 & Banan skall innehålla olika hinder och objekt  &
		1 & $\CheckedBox$ \\
	5 & Man ska kunna klara spelet &
		2 & $\CheckedBox$ \\
	6 & Ett poängsystem, poäng fås av prylar och att döda fiender &
		3 & $\Box$ \\
	7 & Bakgrundsmiljön skall bestå av olika lager för att ge ett djup &
		3 & $\Box$ \\
	8 & En huvudmeny &
		3 & $\Box$ \\
	9 & En startskärm som inviger spelaren i sammanhanget &
		3 & $\Box$ \\
	10 & ”Hit points” &
		3 & $\CheckedBox$ \\
	11 & Extraliv &
		3 & $\Box$ \\
	12 & High scores &
		4 & $\Box$ \\
	13 & En boss skall finnas i slutet av banan &
		4 & $\Box$ \\
	14 & Flera banor &
		5 & $\Box$ \\
	15 & Power-ups: odödlighet, etc. &
		5 & $\Box$ \\
	16 & Musik &
		5 & $\Box$ \\

	\bottomrule
	\end{tabular}
\end{table}

\section{Implementation}
Det här kapitlet behandlar spelets implementation.

\subsection{Abstrakta datatyper}
\begin{adt-table}{\code{map\%} (bana)}{map-adt}
dump-tiles-to-file & filnamn & skickar vidare argumentet till tilemap  \\ 
 
colliding? & obj1, obj2 & returnerar sant om objekten överlappar varandra \\ 
 
colliding-bullets & obj & returnerar alla bullets som kolliderar med obj \\ 
 
colliding-characters & obj & returnerar alla characters som kolliderar med obj \\ 
 
colliding-items & obj & returnerar alla items som kolliderar med obj \\ 
 
colliding-in & lst, obj & returnerar alla element i lst som kolliderar med obj
\end{adt-table}

\subsection{Testning}
Testning kommer att kunna ske kontinuerligt under programmeringen. Hög prioritet ligger på att få till en fungerande spelmiljö, så att testningen kan börja så tidigt som möjligt. Om behov av speciella testfunktioner upptäcks implementeras dessa i stunden.

\subsection{Beskrivning av implementationen}

\section{Utvärdering och erfarenheter}

\section{Tidrapportering}
Vi kommer att rapportera arbetstid fortlöpande i ett separat dokument som finns i Google Drive.


\end{document}