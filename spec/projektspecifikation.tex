\documentclass{scrartcl}
\usepackage[utf8]{inputenc}
\usepackage[swedish]{babel}
\usepackage{booktabs}
\usepackage{longtable}
\usepackage{tabu}
\usepackage{caption}
\usepackage{wasysym}
\usepackage{environ}
\usepackage{array}
\usepackage{graphicx}
\usepackage[table]{xcolor}
\usepackage{alltt}
% utseende på labels
\captionsetup{margin=10pt,font=small,labelfont=bf,labelsep=period}

\author{\textbf{Magnus Johansson}\\
		magjo722@student.liu.se\\ \\
		\textbf{Rolf Lifvergren}\\
		rolli107@student.liu.se}

\title{Projektspecifikation}
\subtitle{Plattformsspel}

%% Kommando för kod inne i texten
%% Användningsexempel:
% \code{(define (square x) (* x x))}
\newcommand{\code}[1]%
{\texttt{#1}}
\newcommand{\filename}[1]%
{\textsl{#1}}

%% Kommando för ADT-tabell
%% Användningsexempel:
% \begin{adt-table}{\texttt{map\%}}{map-adt}
% 	render & canvas, dc & Ritar ut banan på skärmen
% \end{adt-table}
\NewEnviron{adt-table}[2]%
{\rowcolors{2}{gray!25}{white}
\begin{longtabu}{>{\bfseries}lp{4cm}X}
\hiderowcolors
\caption{#1}\label{#2}\\

\toprule
\textbf{Metod} & \textbf{Argument} & \textbf{Beskrivning} \\ 
\midrule
\endfirsthead

\hiderowcolors
\toprule
\multicolumn{3}{c}%
{\bfseries \tablename\ \thetable{}. (forts)} \\
\textbf{Metod} & \textbf{Argument} & \textbf{Beskrivning} \\ 
\midrule
\showrowcolors
\endhead

\hiderowcolors
\multicolumn{3}{r}%
{\bfseries(fortsätter på nästa sida)} \\
\showrowcolors
\endfoot

\bottomrule
\endlastfoot

\BODY
\end{longtabu}}

\begin{document}
\maketitle
\clearpage

\section{Projektplanering}
Detta projekt går ut på att utveckla ett 2D-plattformsspel där spelaren ska hoppa, skjuta, undvika fiender och samla power-ups för att rädda världen (vilket man gör genom att besegra en mäktig boss).

\subsection{Spelhistoria}
Spelet är ett klassiskt 2D-plattformsspel där spelaren styr en karaktär som kan röra sig i en värld genom att springa och hoppa. I världen finns fiender och andra hinder (t.ex. spikar eller avgrunder) som måste undvikas eller besegras med hjälp av olika vapen. För att klara spelet måste en boss, som är den svåraste fienden i spelet, besegras.

Varje gång spelaren blir skadad tappar denne hitpoints. När spelarens hitpoints tagit slut dör karaktären och spelaren förlorar ett liv. När liven tagit slut har spelaren förlorat och måste börja om. 

\subsection{Utvecklingsmetodik}\label{utvecklingsmetodik}
Det är tänkt att vi ska dela upp kodandet lika mellan oss så att arbetet kan färdigställas snabbare. Vi har tänkt att använda oss av Git för versionshantering, till exempel med Astmatix som central server (vilket borde lösa problemet med att koden ska vara tillgänglig även om en gruppmedlem är sjuk eller borta). Vi kommunicerar med varandra via sms, mejl och Skype samt muntligt.

I första hand har vi tänkt att jobba under normal skoltid, dvs 8-17 på vardagar, men kommer även att arbeta på kvällar och helger i de fall tiden inte räcker till på dagarna. Att sitta tillsammans så ofta som möjligt när vi arbetar är önskvärt för att effektivt kunna dela idéer och samarbeta.

Vi räknar med att lägga 160 timmar totalt, vilket motsvarar de 3hp som projektet är på. Det betyder att vi ska jobba 80h/person, vilket i snitt blir 2{,}2 h/dag om vi jobbar 6 dagar i veckan.

\subsection{Grov tidplan}
I vår kravlista ses i vilken ordning vi tänker göra saker. Målsättningen är att krav av prioritet 1 skall göras direkt och prioritet 2 innan halvtidsmötet. Högre ordningens krav kommer efter halvtidsmötet.

Se kapitel \ref{utvecklingsmetodik} för preliminär tidsplan.

\subsection{Betygsambitioner}
Vi vill göra ett så bra projekt som möjligt på den tid vi hinner lägga på projektet och därför satsar vi högt, men eftersom duggorna har gått riktigt bra för oss båda är det inte avgörande för våra betyg i kursen hur högt vi når. Dock är vi båda väldigt intresserade av spelprogrammering så vi kommer med glädje att arbeta med projektet.

\section{Användarmanual}
\subsection{Användarmanual}

Spelet styrs med hjälp av tangentbordet. I tabell \ref{kontroller} ses vilka kommandon som är möjliga. 

\begin{table}[ht]
\caption{Kontroller för att styra spelarkaraktären.}\label{kontroller}
\centering
    \begin{tabular}{ll}
    \toprule
    Tangent  & Funktion \\
    \midrule
    \code{space}, $\uparrow$ , \code{x} & hoppa \\
    $\leftarrow$  & gå vänster \\
    $\rightarrow$ & gå höger \\
    \code{a}, \code{s}     & byta aktuellt vapen \\
    \code{z}        & springa \\
    \code{c}        & skjuta \\
    \bottomrule
    \end{tabular}
\end{table}

För att starta spelet körs filen \textsl{main.scm} i Dr Racket. Spelkaraktären går genast att interagera med när spelet startas. I skärmens övre vänstra hörn syns spelarens head-up-display (HUD). Den innehåller information om hur mycket skada spelaren kan ta emot utan att dö. Dessutom visas vilket vapen som spelaren bär (om något). 

\begin{figure}[h!]
\centering
\includegraphics[width=11cm]{skarmdomp}
\caption{En typisk spelsituation.}\label{screenshot}
\end{figure}

Om spelaren dör visas meddelandet ''Game Over'' och spelet måste startas om för att kunna spelas igen. Målet med spelet är att ta sig till slutet av banan. 

\subsubsection{Level-editor}
Med tangentkombinationen \code{Ctrl-E} aktiveras den inbyggda level-editorn. I den kan användaren redigera hur banans terräng ser ut med hjälp av musen. Texten ''EDITING'' visas i skärmens övre högra hörn när level-editorn är aktiv. När \code{Ctrl-E} trycks ned igen, avaktiveras level-editorn. För mer information om hur level-editorn används, se tabell \ref{editor-kontroller}.

\begin{table}[ht]
\caption{Kontroller för den inbyggda level-editorn.}\label{editor-kontroller}
\centering
    \begin{tabu}{lX}
    \toprule
    Användarinput  & Funktion \\
    \midrule
    \code{Ctrl-E} & Aktivera/inaktivera level-editorn. \\
    \code{Ctrl-S} & Spara den aktuella uppsättningen tiles till hårddisken. \\
    Höger musknapp & Kopiera tilen under muspekaren till att vara aktuell tile. \\
    Vänster musknapp & Sätt ut aktuell tile på positionen under muspekaren. \\
    \bottomrule
    \end{tabu}
\end{table}

\subsection{Kravlista}
Projektets kravlista återfinns i tabell \ref{kravlista} nedan.

\begin{table}[ht]
\rowcolors{2}{gray!25}{white}
	\caption{Kravlista}\label{kravlista}
	\centering
	\begin{tabular}{lp{8cm}rl}
	\toprule
		\# & Beskrivning & Prioritet &  \\
	\midrule
	1 & Spelet ska kunna styras med tangentbordet &
		1 & $\CheckedBox$ \\
	1.A & Spelaren ska kunna hoppa &
		1 & $\CheckedBox$ \\
	1.B & Spelaren ska kunna gå i sidled &
		1 & $\CheckedBox$ \\
	2 & Fiender skall finnas och vara möjliga att interagera med &
		1 & $\CheckedBox$ \\
	3 & Vapen skall kunna användas för att eliminera fiender &
		1 & $\CheckedBox$ \\
	4 & Banan skall innehålla olika hinder och objekt  &
		1 & $\CheckedBox$ \\
	5 & Man ska kunna klara spelet &
		2 & $\CheckedBox$ \\
	6 & Ett poängsystem, poäng fås av prylar och att döda fiender &
		3 & $\Box$ \\
	7 & Bakgrundsmiljön skall bestå av olika lager för att ge ett djup &
		3 & $\Box$ \\
	8 & En huvudmeny &
		3 & $\Box$ \\
	9 & En startskärm som inviger spelaren i sammanhanget &
		3 & $\Box$ \\
	10 & ”Hit points” &
		3 & $\CheckedBox$ \\
	11 & Extraliv &
		3 & $\Box$ \\
	12 & High scores &
		4 & $\Box$ \\
	13 & En boss skall finnas i slutet av banan &
		4 & $\Box$ \\
	14 & Flera banor &
		5 & $\Box$ \\
	15 & Power-ups: odödlighet, etc. &
		5 & $\Box$ \\
	16 & Musik &
		5 & $\Box$ \\

	\bottomrule
	\end{tabular}
\end{table}

\section{Implementation}
Det här kapitlet behandlar spelets implementation.

\subsection{Abstrakta datatyper}
Tabeller \ref{map-adt}--\ref{ring-adt} innehåller specifikation för de abstrakta datatyperna som använts i projektet.

\begin{adt-table}{\code{map\%} (bana)}{map-adt}
dump-tiles-to-file & filnamn & skickar vidare argumentet till tilemap  \\ 
 
colliding? & obj1, obj2 & returnerar sant om objekten överlappar varandra \\ 
 
colliding-bullets & obj & returnerar alla bullets som kolliderar med obj \\ 
 
colliding-characters & obj & returnerar alla characters som kolliderar med obj \\ 
 
colliding-items & obj & returnerar alla items som kolliderar med obj \\ 

colliding-in & lst, obj & returnerar alla element i lst som kolliderar med obj \\ 
 
overlapping-tiles & obj & returnerar en lista med alla koordinater (x . y) för tiles som obj överlappar \\ 
 
colliding-tiles & obj & returnerar \code{\#t} om obj överlappar en solid tile, annars \code{\#f} \\ 
 
get-next-tile-pixel & solid?, x, y, direction & Börjar på position (x,y) och går åt hållet direction tills den stöter på en solid eller tom (beroende på "solid?") tile eller tills banan tar slut. Returnerar x- eller y-koordinaten för den punkten (beroende på om den letade i x- eller y-led)\\ 
 
get-next-solid-pixel & x, y, direction & samma som get-next-tile-pixel fast kollar alltid efter solid tile \\

get-next-empty-pixel & x, y, direction & samma som get-next-tile-pixel fast kollar alltid efter tom tile \\

get-position-tile & x, y & returnerar värdet på en tile vid pixlar (x,y) \\

get-screen-position-tile & x, y & returnerar värdet på en tile vid pixlar (x,y) med x kompenserat för scrollning \\

solid-tile-at? & x, y & returnerar sant om givna pixelkoordinater är del av en solid tile \\

valid-tile-coord? & x, y & returnerar sant om x, y är giltiga koordinater för tiles \\

valid-tile-coord-at-screen? & x, y & returnerar sant om x, y konverterat till tile-koordinater är giltiga koordinater för tiles (x kompenseras för scrollning) \\

set-tile-at-screen! & x, y, value & ändrar tilen vid pixel-koordinater x, y till value med x kompenserat för scrollning \\

add-element! & element & lägger in ett element i banan\\

delete-element! & element & tar bort ett element från banan \\
update & - & uppdaterar alla characters och bullets på banan, uppdaterar positioner, kollar om spelaren har vunnit, justerar scrolled-distance, tar bort bullets som inte ska vara kvar \\
render & canvas, dc & ritar ut allting i banan \\
draw-rectangle & x, y, width, height, color, canvas, dc & ritar ut en rektangel med en viss färg och kompenserar i x-led för sidoscrollning \\
draw-bitmap & bitmap, x, y, canvas, dc & ritar ut en bitmap och kompenserar i x-led för sidoscrollning \\

\end{adt-table}

\begin{adt-table}{\code{character\%} (karaktär)}{karak-adt}
get-position & - & returnerar ett par med karaktärens x- och y-position \\

add- & item & lägger in föremålet i karaktärens inventarium \\

take-weapon! & weapon & gör vapnet aktuellt och lägger in vapnet i inventariet. \\

switch-weapon! & next/prev & roterar inventariet \\

shoot! & - & avfyrar aktuellt vapen \\

remove-self! & - & tar bort karaktären från banan \\

swap-direction & direction & byter karaktärens riktning \\

on-ground? & - & returnerar \code{\#t} om karaktären står på marken \\

find-obstacle & solid?, direction & returnerar en pixel-koordinat för en tom eller solid tile i riktningen direction \\

roof-y & - & Returnerar närmsta solida pixel-koordinat ovan karaktären.  \\

left-x & - & Returnerar närmsta solida pixel-koordinat till vänster om karaktären.  \\

right-x & - & Returnerar närmsta solida pixel-koordinat till höger om karaktären.  \\

ground-y & - & Returnerar närmsta solida pixel-koordinat under karaktären.  \\

decelerate! & - & Minskar karaktärens hastighet i x-led.  \\

gravitate! & - & Ökar karaktärens hastighet i y-led.  \\

push! & dvx, dvy & ändrar karaktärens hastighet i x- och/eller yled \\

jump! & - & Ger karaktären en negativ hastighet i y-led.  \\

render & canvas, dc & säger åt banan att rita en rektangel. \\

move! & - & bestämmer hur spelaren skall röra sig \\

die! & - & se remove-self! \\

hurt! & damage & skadar karaktären \\

update! & - & avgör vad som ska hända varje frame; overridas av \code{player\%} och \code{enemy\%} \\

\end{adt-table}

\begin{adt-table}{\code{tilemap\%} (brickkarta)}{tilemap-adt}

dump-tiles-to-file & filename & spara tiles-vectorn till en fil \\

load-tiles! & input & ladda in tiles från en vector eller en fil \\

valid-tile-coord? & x, y & returnerar \code{\#t} om koordinaterna är tillåtna \\

get-position-tile & x, y & returnerar den tile som finns på x, y \\

get-next-tile-pixel & se map\% & se map\% \\

render-tiles & dc xs,ys & ritar ut tiles på dc \\

render & canvas, dc, scrolled distance & ritar ut alla tiles som ska synas på skärmen \\ 
 
\end{adt-table}


\begin{adt-table}{\code{player\%} (spelare)}{spelar-adt}

set-key! & key, boolean & sätter key till boolean \\

get-key & key & returnerar \code{\#t} om key är nedtryckt \\

update! & - & avgör vad som händer varje frame \\

\end{adt-table}

\begin{adt-table}{\code{enemy\%} (fiende)}{fiende-adt}

left-x & - & returnerar närmsta pixelkoordinaten för antingen en solid tile eller en kant av en platå åt vänster \\

right-x & - & returnerar närmsta pixelkoordinaten för antingen en solid tile eller en kant av en platå åt höger \\

update! & - & avgör vad som händer varje frame \\

\end{adt-table}


\begin{adt-table}{\code{bullet\%} (kula)}{kula-adt}

remove-self & - & tar bort kulan från banan \\

render & canvas, dc & ritar kulan på banan \\

update! & - & kör move! \\

move & - & flyttar kulan \\

\end{adt-table}

\begin{adt-table}{\code{hud\%} (head-up-display)}{hud-adt}

render & canvas, dc & ritar HUD:en på skärmen \\

\end{adt-table}

\begin{adt-table}{\code{weapon\%} (vapen)}{vapen-adt}

render & canvas, dc & ritar vapnet på skärmen  \\

fire! & the-map, direction & avlossar ett skott om vapnet befinner sig hos en spelare \\
 
\end{adt-table}

\begin{adt-table}{Abstrakt datatyp för en dubbelriktad muterbar ring. Procedurerna är inte metoder utan fristående procedurer.}{ring-adt}

make-ring			& -				& skapar en tom ring \\
ring?				& maybe-ring	& returnerar \code{\#t} om maybe-ring är en ring, \code{\#f} annars \\
empty-ring?			& ring			& returnerar \code{\#t} om ring är tom, \code{\#f} annars \\
ring-first			& ring 			& returnerar första elemenetet i ring \\
ring-first-value	& ring 			& returnerar första värdet i ring \\
ring-insert!		& ring, value	& stoppar in ett nytt element med givet värde först i ring. Det gamla första elementet blir ''till höger om'' det nya. \\
ring-rotate-left!	& ring			& roterar ringen ett steg ''åt vänster'' \\
ring-rotate-right!	& ring			& roterar ringen ett steg ''åt höger'' \\

\end{adt-table}

\subsection{Testning}
Testning kommer att kunna ske kontinuerligt under programmeringen. Hög prioritet ligger på att få till en fungerande spelmiljö, så att testningen kan börja så tidigt som möjligt. Om behov av speciella testfunktioner upptäcks implementeras dessa i stunden.

\subsection{Beskrivning av implementationen}
Spelet har programmerats objektorienterat med hjälp av Dr Rackets inbyggda system vilket innebär att spelare, fiender och bana representeras av objekt. Eftersom koden för spelaren och fienderna har många gemensamma drag har en karaktärsklass skapats, som båda dessa bygger på.

När de flesta objekt skapas krävs ett antal initiala argument som kännetecknar det specifika objektet. Dock kan en karaktär skapas utan att skriva något initialt argument eftersom det finns standardvärden som används då inget annat anges, se exempel nedan:

\begin{alltt}
(init-field [x 0]
            [y 0]
            [width 40]
            [height 80]
            [hp 100]
            [the-map #f]
            [bitmap-right #f]
            [bitmap-left #f]
            [direction 'right])
\end{alltt}         

Nyckelordet \code{init-field} sätter parametrarna till de förinställda värdena om inget annat anges av användaren. 

\subsubsection{Allmän kodstruktur}

Koden har delats upp så att varje klass ligger i en egen fil. I filen \filename{main.scm} laddas all annan kod in. Här skapas spelaren, fienderna och banan etc. En timer som tickar 60 ggr/sekund styr hur ofta spellogiken skall uppdateras. 

\subsubsection{Bana}
Banan är en klass som innehåller allting som ska kunna interagera med varandra och det är banan som säger åt allt som finns i den att rita sig genom att sända proceduranropet \code{render}. Det är här det mesta ''kopplas samman'', då det är detta objekt som vet var allting är. Många av procedurerna i \code{map\%} skickas vidare direkt till dess tile-map.



\subsubsection{Tile-map}
En tile-map (''brickkarta'') har skapades för att användas som terräng i banan. Varje tile är 40x40 pixlar och har en viss typ, som representeras av en symbol. Vissa typer av tiles är solida, andra inte. Är den solid krockar karaktärer och skott med den och är den tom kan karaktärer och skott passera igenom den. Mark och hinder består följaktligen av en mängd tiles i olika formationer. Dessa tiles täcker hela banan och kan ha olika utseende genom att de innehåller olika bitmaps (bilder).

I början av spelets utveckling upplevde vi prestandaproblem vid utritning av tiles. Detta löstes genom att inte varje frame rita ut varje enskild tile, utan i stället låta tile-mapen innehålla en bild med alla tiles på hela banan. Denna bild uppdateras varje gång någon tile ändras men är annars statisk. När banan sedan ska ritas ut, kollar spelet vilken del av bilden som kommer att synas och ritar bara ut den delen.

\begin{figure}[h!]
\centering
\includegraphics[width=11cm]{tilekoordinater}
\caption{Ett rutnät av tiles.}\label{tile-rutnat}
\end{figure}

Något som sker väldigt ofta i tile-map är konvertering mellan vanliga pixel-koordinater och tile-koordinater. Tile-mapen kan ses som en hinna som ligger över banan och har sitt eget koordinatsystem, se figur \ref{tile-rutnat}. Eftersom varje tile är 40x40 pixlar ligger tex. punkten (50,50) i den tile som har tile-koordinaterna (1,1). Punkten (0,0) ligger i övre vänstra hörnet av skärmen, och båda koordinatsystemen utgår från övre vänstra hörnet. För att räkna ut vilken tile en viss koordinat ligger i tas därför hänsyn till en tiles mått. Dessutom avrundas de erhållna koordinaterna till heltal med hjälp av proceduren \code{inexact->exact} för att kunna appliceras till en befintlig tile-koordinat, se exempelkod nedan.

\begin{alltt}
(define/public (get-tile-coord-pos x y) 
      (values (inexact->exact (floor (/ x tile-size)))
              (inexact->exact (floor (/ y tile-size)))))
\end{alltt}         


\subsubsection{Karaktärer}

Karaktärsklassen är gemensam för fiender och spelare. Denna klass innehåller metoder som hanterar vad som ska hända när karaktären krockar med något, tar skada, tar upp och byter vapen etc.

Metoderna \code{left-x} och \code{right-x} användes som sagt till att finna nästa solida tile eller kant av plattform i en viss riktning. För spelaren så har de endast gällt för solida tiles medan de för fiender gäller båda. Detta beror på att fiender ska ändra gångriktning både när de stöter i ett hinder och när de kommer fram till en kant. En lösning på detta var att ''overrida'' dessa procedurer som ju ärvts från \code{character\%} i \code{enemy\%}.

\code{ground-y} användes för att hålla reda på var marken är. I koden så varierar markkoordinaterna beroende på var spelaren befinner sig eftersom marken definieras som närmsta solida tile under spelaren. Detta innebär att om man hoppar från en plattform mot en högre så att man befinner sig ovanför den högre plattformen så sätts den övre plattformens koordinater till markkoordinater. 



\subsubsection{vapen}

Ett vapen initieras med en bullet, som sedan skapas när vapnet avfyras. Varje vapen skapas dessutom med en egen cooldown, så att skott kan skjutas olika frekvent med olika vapen. 

\subsubsection{Skott}

Skotten är objekt som kan avlossas från vapen. Dessa tas bort när de åker utanför den synliga delen av banan för att spara datorkapacitet. De tas även bort vid krock med solida tiles samt efter att ha träffat tex. en fiende. 

Skotten åker åt det håll som spelaren är riktad åt. Detta har implementerats genom att skottet får en direction när det skapas som är likriktad vapnets bärare, dvs karaktären. 

\subsubsection{Dubbelriktad muterbar ring}

En dubbelriktad muterbar ring implementerades som en cirkulär struktur av muterbara cons-celler. Denna datatyp användPs sedan i \code{character\%} för att representera vilka vapen en viss karaktär hade i sin ägo. På så sätt kan spelaren på ett enkelt sätt växla mellan sina vapen både framåt och bakåt utan att någonsin nå ett ''slut'' på vapenlistan.

\subsubsection{Level-editor}
För att underlätta skapandet av en spelbar bana konstruerades även en enkel level-editor som styrs med musen och tangentbordet. Med hjälp av level-editorn kan användaren spara den aktuella banans terräng (fiender och vapen inte inräknade) till hårddisken.

Denna skapades främst som ett verktyg för att utveckla spelet och kräver i den nuvarande implementationen även ändringar i koden för att t.ex. sätta ut fiender och vapen, men skulle med små modifikationer kunna användas av vilken användare som helst för att utveckla helt nya nivåer från grunden.

När spelet startas laddas alla tiles på banan in från filen \filename{level.sexpr}, vilken också är den fil som level-editorn skriver till.

\section{Utvärdering och erfarenheter}

Projektet har gått bra överlag och vi är nöjda med resultatet. Även om det vi skapat inte är ett fullfjädrat plattformsspel med invecklade banor och många olika fiender, har vi åstadkommit en relativt kompetent spelmotor för 2D-plattformsspel som lämnar mycket utrymme för utbyggnad.

Vi har inte lagt ned riktigt lika mycket tid som vi hade räknat med, ungefär 20 timmar mindre per person. Dock skulle vi inte vilja påstå att vi har lagt ner för lite tid, eftersom vi inte hade hunnit jobba så mycket mer på den utsatta tiden om vi inte hade suttit ännu mer på kvällar och helger. Arbetsfördelningen var relativt jämn, dock hann Rolf göra mer än Magnus på samma antal timmar arbete eftersom han är en mer erfaren programmerare. Vi jobbade tillsammans ganska länge innan vi delade upp kodandet, vilket kan ha medfört att vi i slutändan hann mindre totalt. Dock kändes det tryggt att bygga upp grundstommen tillsammans. 

Vi har i princip använt specifikationen till att titta på vår kravlista. Om vi hade funderat mer på hur vi tänkte lägga upp kodandet innan vi satte igång, och skrivit in våra tankar, hade vi förmodligen tittat mer på den, men vi båda kände att vi ville komma igång och använda oss av arbetssättet att fundera under arbetets gång vad som kunde behövas osv. Vi ville absolut inte låsa oss till vissa grejer (t.ex. bestämma precis vilka kontroller som skulle användas och hur karaktärerna skulle se ut) utan ville känna friheten att kunna ändra oss efterhand. 

Det hade varit bra att få mer information om hur projektet var tänkt att gå till i förväg. Det finns en pdf på kurshemsidan från 2012 (''Breddföreläsning om ADT:er, OOP, versionshantering, m.m.'') med information som vi gärna hade sett i form av en föreläsning även i år. Det hade nog hjälpt många (inklusive oss) att komma igång ordentligt med arbetet med versionshantering och så vidare.

Vi har bland annat lärt oss att dela upp kodandet och samarbeta via versionhanteringssystemet git. Dessutom har ytterligare erfarenheter erhållits inom programmering i allmänhet och objektorientering i synnerhet. 

Vi har dessutom lärt oss att arbeta i \LaTeX\ för att skriva rapporter.

\section{Tidrapportering}
Rapportering av arbetstid har skett fortlöpande i ett separat ods-dokument som finns bifogat.

\end{document}
